\documentclass[12pt,origdate]{lettre}%
\usepackage[utf8x]{inputenc}
\usepackage[french]{babel}
%\usepackage[T1]{fontenc}
%\usepackage{mltex}

%-------------------------------------------------------------------------
%   Partie sur le. procurant.e
%-------------------------------------------------------------------------


% Mettre ici le prenom de la personne faisant la procuration
% et entre les {} le nom de famille
\newcommand{\ProcurantId}{ {{- cleaned_data["firstname"] -}} {} \textsc{ {{- cleaned_data["lastname"] -}} }}

% Mettre ici la suite des pr\'{e}noms de la personne faisant la procuration
% et entre les {} le nom de famille
\newcommand{\ProcurantFullId}{ {{- cleaned_data["listofname"] -}} {} \textsc{ {{- cleaned_data["lastname"] -}} } }

% Mettre ici le num\'{e}ro de t\'{e}l\'{e}phone de la personnne faisant la
% procuration
\newcommand{\ProcurantTel}{ {{- cleaned_data["telephone"] -}} }

% Mettre ici le num\'{e}ro, le type de voie et le nom de la voie
% de la personne faisant la procuration
\newcommand{\ProcurantAddresseRue}{ {{- cleaned_data["address1"] -}} }

% Mettre ici le code postal (les 2 premiers chiffres avant le ~)
% et entre {} le nom de la ville
\newcommand{\ProcurantAddresseVille}{ {{- cleaned_data["address2"] -}} }

% Ne pas toucher
\newcommand{\ProcurantAddressFull}{\ProcurantId\\\ProcurantAddresseRue\\\ProcurantAddresseVille}

% Mettre entre {} le nom de la ville où est r\'{e}dig\'{e}e la procuration
\newcommand{\ProcurantLocation}{\textsc{ {{- cleaned_data["location"] -}} }}

% Mettre ici l'adresse mail de la personne faisant la procuration
\newcommand{\ProcurantEmail}{ {{- cleaned_data["email"] -}} }

% Mettre ici le nom du d\'{e}partement où habites la personne
% cela est utilis\'{e} pour la caisse de CPAM dont elle d\'{e}pend
\newcommand{\ProcurantDepartement}{ {{- cleaned_data["departement"] -}} }

% Mettre ici la date de naissance de la personne faisant la procuration
\newcommand{\ProcurantDOB}{ {{- cleaned_data["dob"] -}} }

% Mettre ici la ville de naissance et entre () le d\'{e}partement de
% naissance de la personne faisant la procuration
\newcommand{\ProcurantPOB}{ {{- cleaned_data["pob"] -}} }

% Malheureusement, pour des raisons d'idenfication, il faut mettre
% le deadname de la personne faisant la procuration et son nom de
% famille
\newcommand{\ProcurantOldId}{ {{- cleaned_data["deadname"] -}} {} \textsc{ {{- cleaned_data["lastname"] -}} }}

% Mettre ici le num\'{e}ro de s\'{e}curit\'{e} sociale de la personne faisant
% la procuration
\newcommand{\ProcurantSecu}{ {{- cleaned_data["ss"] -}} }

% Mettre ici le type d'accord souhait\'{e} pour la personne faisant
% la procuration.
% 0 = femme trans
% 1 = homme trans
\newcommand{\ProcurantGender}{ {{- cleaned_data["gender"] -}} } % 0 = femme, 1 = homme


% Date de fin de la procuration
\newcommand{\Date}{ {{- cleaned_data["date"] -}} }



% Ne pas toucher
\newcommand{\Civilite}{
    En effet, comme l'a rappel\'{e} de nombreuses fois le D\'{e}fenseur des Droits dans des rapports, synth\`{e}ses, recommandations et d\'{e}cisions, la civilit\'{e} est une affaire d'usage et n'est aucunement li\'{e}e \`{a} la mention de sexe \`{a} l'\'{E}tat-Civil. Je vous engage \`{a} vous renseigner plus avant sur le site du D\'{e}fenseur des Droits si le moindre doute persiste.

    \medskip

    De surcro\^{\i}t, il a not\'{e} qu'une persistence d'une civilit\'{e} erronn\'{e}e, c'est-\`{a}-dire non conforme \`{a} l'identit\'{e} de genre r\'{e}elle d'une personne transgenre, relevait du harc\`{e}lement discriminatoire bas\'{e} sur l'identit\'{e} de genre.

    \medskip

    Pour toutes ces raisons, je vous demande de mettre \`{a} jour, dans les plus brefs d\'{e}lais, mon pr\'{e}nom et ma civilit\'{e} dans vos registres, et de me tenir inform\'{e}e par retour de courrier \'{e}lectronique de la bonne prise en compte de ces informations.
}

\begin{document}
%-----------------------------------------------%
% Le blabla pas int\'{e}ressant                     %
%-----------------------------------------------%
\begin{letter}{\`{A} l'attention de la CPAM \ProcurantDepartement}

    \date{le \Date}
    \name{\ProcurantId}
    \address{\centering\ProcurantAddressFull}
    \telephone{\ProcurantTel}
    \nofax
    \lieu{\ProcurantLocation}
    \signature{\ProcurantId}
    \email{\ProcurantEmail}

\conc{Demande de changement de pr\'{e}nom et de civilit\'{e} d'une personne transgenre}
    \opening{Madame, Monsieur,}
%-----------------------------------------------%
% Le corps de la lettre de motivation           %
% \`{a} vous de jouer !                             %
%-----------------------------------------------%
%-% \`{\`{a}} remplacer par votre lettre de motivation
    Je suis \ProcurantFullId et j'ai r\'{e}cemment obtenu mon changement de pr\'{e}nom \`{a} l'\'{E}tat-Civil et je souhaite effectuer les d\'{e}marches de changement de pr\'{e}nom et de civilit\'{e} dans les registres de la CPAM \ProcurantDepartement.

    \medskip

    Je suis \if\ProcurantGender0 connue \else connu \fi de vos services sous le nom de \ProcurantOldId~ et j'ai le num\'{e}ro de S\'{e}curit\'{e} Sociale \ProcurantSecu.

    \medskip

    En vertu de la d\'{e}cision de changement de pr\'{e}nom d\'{e}livr\'{e}e par l'\'{E}tat-Civil, je vous prie de bien vouloir changer mon pr\'{e}nom dans vos registres.

    \medskip

    De plus, je vous prie de bien vouloir changer ma civilit\'{e} de \if\ProcurantGender0 \og Monsieur \fg{} \`{a} \og Madame \fg{} \else \og Madame \fg{} \`{a} \og Monsieur \fg{} \fi dans vos registres, et ce pr\'{e}alablement \`{a} toute d\'{e}cision de changement de mention de sexe \`{a} l'\'{E}tat-Civil, puisque je suis une personne transgenre.

    \medskip

    \Civilite
%-----------------------------------------------%
% Le blabla pas int\'{e}ressant                     %
%-----------------------------------------------%
    \closing{Je vous prie de bien vouloir croire, Madame, Monsieur, en l'assurance de ma consid\'{e}ration,}
%-% \cc{Listes des autres destinataires}
    \encl{Copie des pi\`{e}ces d'identit\'{e} de \ProcurantId {} et d\'{e}cision de changement de pr\'{e}nom de \ProcurantId}

\end{letter}
\end{document}
