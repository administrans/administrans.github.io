\documentclass[12pt,origdate]{lettre}%
\usepackage[utf8x]{inputenc}
\usepackage[french]{babel}
%\usepackage[T1]{fontenc}
%\usepackage{mltex}

%-------------------------------------------------------------------------
%   Partie sur le. procurant.e
%-------------------------------------------------------------------------


% Mettre ici le prenom de la personne faisant la procuration
% et entre les {} le nom de famille
\newcommand{\ProcurantId}{ {{- cleaned_data["procurantfirstname"] -}} {} \textsc{ {{- cleaned_data["procurantlastname"] -}} }}

% Mettre ici la suite des pr\'{e}noms de la personne faisant la procuration
% et entre les {} le nom de famille
\newcommand{\ProcurantFullId}{ {{- cleaned_data["procurantlistofname"] -}} {} \textsc{ {{- cleaned_data["procurantlastname"] -}} } }


% Mettre ici le num\'{e}ro, le type de voie et le nom de la voie
% de la personne faisant la procuration
\newcommand{\ProcurantAddresseRue}{ {{- cleaned_data["procurantaddress1"] -}} }

% Mettre ici le code postal (les 2 premiers chiffres avant le ~)
% et entre {} le nom de la ville
\newcommand{\ProcurantAddresseVille}{ {{- cleaned_data["procurantaddress2"] -}} }

% Ne pas toucher
\newcommand{\ProcurantAddressFull}{\ProcurantId\\\ProcurantAddresseRue\\\ProcurantAddresseVille}


% Mettre ici la date de naissance de la personne faisant la procuration
\newcommand{\ProcurantDOB}{ {{- cleaned_data["procurantdob"] -}} }

% Mettre ici la ville de naissance et entre () le d\'{e}partement de
% naissance de la personne faisant la procuration
\newcommand{\ProcurantPOB}{ {{- cleaned_data["procurantpob"] -}} }

% Malheureusement, pour des raisons d'idenfication, il faut mettre
% le deadname de la personne faisant la procuration et son nom de
% famille
\newcommand{\ProcurantOldId}{ {{- cleaned_data["procurantdeadname"] -}} {} \textsc{ {{- cleaned_data["procurantlastname"] -}} }}

% Mettre ici le num\'{e}ro de s\'{e}curit\'{e} sociale de la personne faisant
% la procuration
%\newcommand{\ProcurantSecu}{ {{- cleaned_data["procurantss"] -}} }


% Mettre ici le type d'accord souhait\'{e} pour la personne faisant
% la procuration.
% 0 = femme trans
% 1 = homme trans
\newcommand{\ProcurantGender}{ {{- cleaned_data["procurantgender"] -}} } % 0 = femme, 1 = homme


\newcommand{\Date}{ {{- cleaned_data["date"] -}} }

\newcommand{\ProcurantVille}{ {{- cleaned_data["procurantville"] -}} }
\newcommand{\PersonIgnoreDeadname}{ {{- cleaned_data["personignoredeadname"] -}} }
% Mettre ici le nom de l'\'{e}cole/universit\'{e}/autre \`{a} laquelle la demande
% de rectification et d'information est demand\'{e}e
%\newcommand{\ProcurantEcole}{ {{- cleaned_data["procurantecole"] -}} }

%-------------------------------------------------------------------------
%   Partie sur le. procur\'{e}.e
%-------------------------------------------------------------------------

% Mettre ici le pr\'{e}nom et le nom de famille entre {} de la personne
% faisant la d\'{e}marche
\newcommand{\PersonId}{ {{- cleaned_data["personfirstname"] -}} {} \textsc{ {{- cleaned_data["personlastname"] -}} }}

% Mettre ici tous les pr\'{e}noms de la personne faisant la d\'{e}marche et
% le nom de famille entre {}
\newcommand{\PersonFullId}{ {{- cleaned_data["personlistofname"] -}} {} \textsc{ {{- cleaned_data["personlastname"] -}} }}

% Mettre ici le num\'{e}ro de t\'{e}l\'{e}phone
\newcommand{\PersonTel}{ {{- cleaned_data["persontelephone"] -}} }

% Mettre ici : num\'{e}ro de la voie, type de voie, nom de la voie
\newcommand{\PersonAddresseRue}{ {{- cleaned_data["personaddress1"] -}} }

% Mettre ici le code postal et la ville
\newcommand{\PersonAddresseVille}{ {{- cleaned_data["personaddress2"] -}} }

% Ne pas toucher
\newcommand{\PersonAddressFull}{\PersonId\\\PersonAddresseRue\\\PersonAddresseVille}

% Mettre ici le nom de la ville d'où est \'{e}crit le courrier
\newcommand{\PersonLocation}{\textsc{ {{- cleaned_data["personlocation"] -}} }}

% Mettre ici le mail de la personne faisant les d\'{e}marches
\newcommand{\PersonEmail}{ {{- cleaned_data["personemail"] -}} }

% Mettre ici la date de naissance de la personne faisant les d\'{e}marches
\newcommand{\PersonDOB}{ {{- cleaned_data["persondob"] -}} }

% Mettre ici la ville et d\'{e}partement de naissance de la personne faisant
% les d\'{e}marches
\newcommand{\PersonPOB}{ {{- cleaned_data["personpob"] -}} }

% Pour les accords
% 0 = feminin
% 1 = masculun
\newcommand{\PersonGender}{ {{- cleaned_data["persongender"] -}} } % 0 = femme, 1 = homme


\begin{document}
%-----------------------------------------------%
% Le blabla pas int\'{e}ressant                     %
%-----------------------------------------------%
\begin{letter}{\`{A} l'attention du service d'\'{E}tat-Civil de \PersonLocation}

    \date{le \Date}
    \name{\PersonId}
    \address{\centering\PersonAddressFull}
    \telephone{\PersonTel}
    \nofax
    \lieu{\ProcurantVille}
    \signature{\PersonId}
    \email{\PersonEmail}

\conc{Attestation de changement de pr\'{e}nom pour une personne transgenre}
    \opening{Madame, Monsieur,}
%-----------------------------------------------%
% Le corps de la lettre de motivation           %
% \`{a} vous de jouer !                             %
%-----------------------------------------------%
%-% \`{a} remplacer par votre lettre de motivation
    Je\if\PersonGender0\ sous-sign\'{e}e \else\ sous-sign\'{e} \fi\PersonFullId{},\if\PersonGender0\ n\'{e}e \else\ n\'{e} \fi le \PersonDOB{} \`{a} \PersonPOB, demeurant au \PersonAddresseRue{} \PersonAddresseVille{} d\'{e}clare sur l'honneur \if\PersonIgnoreDeadname0
    ignorer le pr\'{e}nom d'\'{E}tat-Civil de \ProcurantFullId{},
    \if\ProcurantGender0 n\'{e}e \else n\'{e} \fi le \ProcurantDOB{} \`{a} \ProcurantPOB{} et demeurant \`{a} \ProcurantAddresseRue{} \ProcurantAddresseVille{} et ne l'appeler que \ProcurantId
    \else n'appeler \ProcurantOldId
    \if\ProcurantGender0 n\'{e}e \else n\'{e} \fi le \ProcurantDOB{} \`{a} \ProcurantPOB{} et demeurant \`{a} \ProcurantAddresseRue{} \ProcurantAddresseVille{} que par son v\'{e}ritable pr\'{e}nom : \ProcurantFullId\fi.
%-----------------------------------------------%
% Le blabla pas int\'{e}ressant                     %
%-----------------------------------------------%
    \closing{Fait pour faire valoir ce que de droit,}
%-% \cc{Listes des autres destinataires}
    \encl{Copie d'une pi\`{e}ce d'identit\'{e} de \PersonId}
\end{letter}
\end{document}
