\documentclass[12pt,origdate]{lettre}%
\usepackage[utf8x]{inputenc}
\usepackage[french]{babel}
%\usepackage[T1]{fontenc}
%\usepackage{mltex}

%-------------------------------------------------------------------------
%   Partie sur le. procurant.e
%-------------------------------------------------------------------------


% Mettre ici le prenom de la personne faisant la procuration
% et entre les {} le nom de famille
\newcommand{\ProcurantId}{ {{- form.cleaned_data["procurantfirstname"] -}} {} \textsc{ {{- form.cleaned_data["procurantlastname"] -}} }}

% Mettre ici la suite des pr\'{e}noms de la personne faisant la procuration
% et entre les {} le nom de famille
\newcommand{\ProcurantFullId}{ {{- form.cleaned_data["procurantlistofname"] -}} {} \textsc{ {{- form.cleaned_data["procurantlastname"] -}} } }

% Mettre ici le num\'{e}ro de t\'{e}l\'{e}phone de la personnne faisant la
% procuration
\newcommand{\ProcurantTel}{ {{- form.cleaned_data["procuranttelephone"] -}} }

% Mettre ici le num\'{e}ro, le type de voie et le nom de la voie
% de la personne faisant la procuration
\newcommand{\ProcurantAddresseRue}{ {{- form.cleaned_data["procurantaddress1"] -}} }

% Mettre ici le code postal (les 2 premiers chiffres avant le ~)
% et entre {} le nom de la ville
\newcommand{\ProcurantAddresseVille}{ {{- form.cleaned_data["procurantaddress2"] -}} }

% Ne pas toucher
\newcommand{\ProcurantAddressFull}{\ProcurantId\\\ProcurantAddresseRue\\\ProcurantAddresseVille}

% Mettre entre {} le nom de la ville où est r\'{e}dig\'{e}e la procuration
\newcommand{\ProcurantLocation}{\textsc{ {{- form.cleaned_data["procurantlocation"] -}} }}

% Mettre ici l'adresse mail de la personne faisant la procuration
\newcommand{\ProcurantEmail}{ {{- form.cleaned_data["procurantemail"] -}} }

% Mettre ici le nom du d\'{e}partement où habites la personne
% cela est utilis\'{e} pour la caisse de CPAM dont elle d\'{e}pend
%\newcommand{\ProcurantDepartement}{ {{- form.cleaned_data["procurantdepartement"] -}} }

% Mettre ici la date de naissance de la personne faisant la procuration
\newcommand{\ProcurantDOB}{ {{- form.cleaned_data["procurantdob"] -}} }

% Mettre ici la ville de naissance et entre () le d\'{e}partement de
% naissance de la personne faisant la procuration
\newcommand{\ProcurantPOB}{ {{- form.cleaned_data["procurantpob"] -}} }

% Malheureusement, pour des raisons d'idenfication, il faut mettre
% le deadname de la personne faisant la procuration et son nom de
% famille
\newcommand{\ProcurantOldId}{ {{- form.cleaned_data["procurantdeadname"] -}} {} \textsc{ {{- form.cleaned_data["procurantlastname"] -}} }}

% Mettre ici le num\'{e}ro de s\'{e}curit\'{e} sociale de la personne faisant
% la procuration
%\newcommand{\ProcurantSecu}{ {{- form.cleaned_data["procurantss"] -}} }

% Mettre ici le nom de la ville dont la personne faisant la
% procuration d\'{e}pend d'un point de vue des imp\^{o}ts
%\newcommand{\ProcurantImpots}{Ville du centre des impots}

% Mettre ici le num\'{e}ro fiscal de la personne faisant la procuration
%\newcommand{\ProcurantFiscal}{num\'{e}ro fiscal}

% Mettre ici le type d'accord souhait\'{e} pour la personne faisant
% la procuration.
% 0 = femme trans
% 1 = homme trans
\newcommand{\ProcurantGender}{ {{- form.cleaned_data["procurantgender"] -}} } % 0 = femme, 1 = homme

% Mettre ici le nom de l'\'{e}cole/universit\'{e}/autre \`{a} laquelle la demande
% de rectification et d'information est demand\'{e}e
\newcommand{\ProcurantEcole}{ {{- form.cleaned_data["procurantecole"] -}} }

% Mettre ici le nom de la banque de la personne faisant la procuration
%\newcommand{\ProcurantBanque}{Nom de la banque}

% Mettre ici le nom de l'entreprise \`{a} laquelle la personne faisant
% la procuration veut faire un changement
%\newcommand{\ProcurantEntreprise}{Nom de l'entreprise}

% Mettre ici le num\'{e}ro de contrat de l'entreprise cit\'{e}e au dessus
%\newcommand{\ProcurantContrat}{Num\'{e}ro de contrat}

% Date de fin de la procuration
\newcommand{\FinProc}{ {{- form.cleaned_data["finprocuration"] -}} }
\newcommand{\DebutProc}{ {{- form.cleaned_data["debutprocuration"] -}} }
% Date d'envoi du premier courrier \`{a} Free (utile pour la relance)
\newcommand{\DateCourrierFree}{XX mois XXXX}

% Date d'envoi du premier courrier aux Imp\^{o}ts (utile pour la relance)
\newcommand{\DateCourrierImpots}{XX mois XXXX}

% Date d'envoi du premier courrier \`{a} la CPAM (utile pour la relance)
\newcommand{\DateCourrierCPAM}{XX mois XXXX}

% Date d'envoi du premier courrier \`{a} l'\'{e}cole (utile pour la relance)
\newcommand{\DateCourrierEcole}{XX mois XXXX}

% Date d'envoi du premier courrier \`{a} la banque (utile pour la relance)
\newcommand{\DateCourrierBanque}{XX mois XXXX}

% Date d'envoi du premier courrier \`{a} l'entreprise (utile pour la relance)
\newcommand{\DateCourrierEntreprise}{XX mois XXXX}
%-------------------------------------------------------------------------
%   Partie sur le. procur\'{e}.e
%-------------------------------------------------------------------------

% Mettre ici le pr\'{e}nom et le nom de famille entre {} de la personne
% faisant la d\'{e}marche
\newcommand{\PersonId}{ {{- form.cleaned_data["personfirstname"] -}} {} \textsc{ {{- form.cleaned_data["personlastname"] -}} }}

% Mettre ici tous les pr\'{e}noms de la personne faisant la d\'{e}marche et
% le nom de famille entre {}
\newcommand{\PersonFullId}{ {{- form.cleaned_data["personlistofname"] -}} {} \textsc{ {{- form.cleaned_data["personlastname"] -}} }}

% Mettre ici le num\'{e}ro de t\'{e}l\'{e}phone
\newcommand{\PersonTel}{ {{- form.cleaned_data["persontelephone"] -}} }

% Mettre ici : num\'{e}ro de la voie, type de voie, nom de la voie
\newcommand{\PersonAddresseRue}{ {{- form.cleaned_data["personaddress1"] -}} }

% Mettre ici le code postal et la ville
\newcommand{\PersonAddresseVille}{ {{- form.cleaned_data["personaddress2"] -}} }

% Ne pas toucher
\newcommand{\PersonAddressFull}{\PersonId\\\PersonAddresseRue\\\PersonAddresseVille}

% Mettre ici le nom de la ville d'où est \'{e}crit le courrier
\newcommand{\PersonLocation}{\textsc{ {{- form.cleaned_data["personlocation"] -}} }}

% Mettre ici le mail de la personne faisant les d\'{e}marches
\newcommand{\PersonEmail}{ {{- form.cleaned_data["personemail"] -}} }

% Mettre ici la date de naissance de la personne faisant les d\'{e}marches
\newcommand{\PersonDOB}{ {{- form.cleaned_data["persondob"] -}} }

% Mettre ici la ville et d\'{e}partement de naissance de la personne faisant
% les d\'{e}marches
\newcommand{\PersonPOB}{ {{- form.cleaned_data["personpob"] -}} }

% Pour les accords
% 0 = feminin
% 1 = masculun
\newcommand{\PersonGender}{ {{- form.cleaned_data["persongender"] -}} } % 0 = femme, 1 = homme



% Ne pas toucher
\newcommand{\Civilite}{
    En effet, comme l'a rappel\'{e} de nombreuses fois le D\'{e}fenseur des Droits dans des rapports, synth\`{e}ses, recommandations et d\'{e}cisions, la civilit\'{e} est une affaire d'usage et n'est aucunement li\'{e}e \`{a} la mention de sexe \`{a} l'\'{E}tat-Civil. Je vous engage \`{a} vous renseigner plus avant sur le site du D\'{e}fenseur des Droits si le moindre doute persiste.

    \medskip

    De surcro\^{\i}t, il a not\'{e} qu'une persistence d'une civilit\'{e} erronn\'{e}e, c'est-\`{a}-dire non conforme \`{a} l'identit\'{e} de genre r\'{e}elle d'une personne transgenre, relevait du harc\`{e}lement discriminatoire bas\'{e} sur l'identit\'{e} de genre.

    \medskip

    Pour toutes ces raisons, je vous demande de mettre \`{a} jour, dans les plus brefs d\'{e}lais, le pr\'{e}nom et la civilit\'{e} de \ProcurantId {} dans vos registres, et de me tenir inform\'{e}e par retour de courrier \'{e}lectronique de la bonne prise en compte de ces informations.
}

\begin{document}
%-----------------------------------------------%
% Le blabla pas int\'{e}ressant                     %
%-----------------------------------------------%
\begin{letter}{\`{A} l'attention du service scolarit\'{e} de \ProcurantEcole}

    \date{le \DebutProc}
    \name{\ProcurantId}
    \address{\centering\ProcurantAddressFull}
    \telephone{\ProcurantTel}
    \nofax
    \lieu{\ProcurantLocation}
    \signature{\ProcurantId}
    \email{\ProcurantEmail}

    \conc{Procuration}

    \opening{Madame, Monsieur,}
%-----------------------------------------------%
% Le corps de la lettre de motivation           %
% \`{a} vous de jouer !                             %
%-----------------------------------------------%
%-% remplacer par votre lettre de motivation
    Je \if\ProcurantGender0 sous-sign\'{e}e \else sous-sign\'{e} \fi \ProcurantFullId, \if\ProcurantGender0 n\'{e}e \else n\'{e} \fi le \ProcurantDOB {} \`{a} \ProcurantPOB, demeurant au \ProcurantAddresseRue {} \ProcurantAddresseVille agissant en tant que mandant d\'{e}clare donner pouvoir \`{a} la pr\'{e}sente \`{a} \PersonFullId, \if\PersonGender0 n\'{e}e \else n\'{e} \fi le \PersonDOB \`{a} \PersonPOB, demeurant au \PersonAddresseRue {} \PersonAddresseVille aupr\`{e}s du service scolarit\'{e} de \ProcurantEcole.

    \medskip

    Par cette procuration valable de \DebutProc jusqu'au \FinProc, le mandataire est en droit d'effectuer toutes les d\'{e}marches n\'{e}cessaire afin de faire valoir mon changement de pr\'{e}nom et de civilit\'{e}.
%-----------------------------------------------%
% Le blabla pas int\'{e}ressant                     %
%-----------------------------------------------%
    \closing{Pour faire valoir ce que de droit,}
%-% \cc{Listes des autres destinataires}
    \encl{Copie des pi\`{e}ces d'identit\'{e} de \ProcurantId et de \PersonId}
\end{letter}



\begin{letter}{\`{A} l'attention du service scolarit\'{e} de \ProcurantEcole}

    \date{le \DebutProc}
    \name{\PersonId}
    \address{\centering\PersonAddressFull}
    \telephone{\PersonTel}
    \nofax
    \lieu{\PersonLocation}
    \signature{\PersonId}
    \email{\PersonEmail}

\conc{Demande de changement de pr\'{e}nom et de civilit\'{e} d'une personne transgenre}
    \opening{Madame, Monsieur,}
%-----------------------------------------------%
% Le corps de la lettre de motivation           %
% \`{a} vous de jouer !                             %
%-----------------------------------------------%
%-% \`{\`{a}} remplacer par votre lettre de motivation
    Je suis \if\PersonGender0 mandat\'{e}e \else mandat\'{e} \fi par \ProcurantFullId qui a r\'{e}cemment obtenu son changement de pr\'{e}nom \`{a} l'\'{E}tat-Civil afin d'effectuer les d\'{e}marches de changement de pr\'{e}nom et de civilit\'{e} dans les registres du service scolarit\'{e} de \ProcurantEcole.

    \medskip

    \ProcurantId, est \if\ProcurantGender0 connue \else connu \fi de vos services sous le nom de \ProcurantOldId.

    \medskip

    En vertu de la d\'{e}cision de changement de pr\'{e}nom d\'{e}livr\'{e}e par l'\'{E}tat-Civil, je vous prie de bien vouloir changer son pr\'{e}nom dans vos registres.

    \medskip

    De plus, je vous prie de bien vouloir changer sa civilit\'{e} de \if\ProcurantGender0 \og Monsieur \fg{} \`{a} \og Madame \fg{} \else \og Madame \fg{} \`{a} \og Monsieur \fg{} \fi dans vos registres, et ce pr\'{e}alablement \`{a} toute d\'{e}cision de changement de mention de sexe \`{a} l'\'{E}tat-Civil, puisque qu'il s'agit d'une personne transgenre.

    \medskip

    \Civilite

    \medskip

    Enfin, je vous prie de bien vouloir m'indiquer quelle proc\'{e}dure \ProcurantId{} doit effectuer afin d'obtenir une version \'{a} jour, en terme de pr\'{e}nom et de civilit\'{e}, de ses dipl\^{o}mes.
%-----------------------------------------------%
% Le blabla pas int\'{e}ressant                     %
%-----------------------------------------------%
    \closing{Je vous prie de bien vouloir croire, Madame, Monsieur, en l'assurance de ma consid\'{e}ration,}
%-% \cc{Listes des autres destinataires}
    \encl{Copie des pi\`{e}ces d'identit\'{e} de \ProcurantId, \PersonId, procuration de \ProcurantId et d\'{e}cision de changement de pr\'{e}nom de \ProcurantId}

\end{letter}
\end{document}
